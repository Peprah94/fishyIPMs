% Options for packages loaded elsewhere
\PassOptionsToPackage{unicode}{hyperref}
\PassOptionsToPackage{hyphens}{url}
\PassOptionsToPackage{dvipsnames,svgnames,x11names}{xcolor}
%
\documentclass[
]{article}

\usepackage{amsmath,amssymb}
\usepackage{iftex}
\ifPDFTeX
  \usepackage[T1]{fontenc}
  \usepackage[utf8]{inputenc}
  \usepackage{textcomp} % provide euro and other symbols
\else % if luatex or xetex
  \usepackage{unicode-math}
  \defaultfontfeatures{Scale=MatchLowercase}
  \defaultfontfeatures[\rmfamily]{Ligatures=TeX,Scale=1}
\fi
\usepackage{lmodern}
\ifPDFTeX\else  
    % xetex/luatex font selection
\fi
% Use upquote if available, for straight quotes in verbatim environments
\IfFileExists{upquote.sty}{\usepackage{upquote}}{}
\IfFileExists{microtype.sty}{% use microtype if available
  \usepackage[]{microtype}
  \UseMicrotypeSet[protrusion]{basicmath} % disable protrusion for tt fonts
}{}
\usepackage{xcolor}
\setlength{\emergencystretch}{3em} % prevent overfull lines
\setcounter{secnumdepth}{5}
% Make \paragraph and \subparagraph free-standing
\ifx\paragraph\undefined\else
  \let\oldparagraph\paragraph
  \renewcommand{\paragraph}[1]{\oldparagraph{#1}\mbox{}}
\fi
\ifx\subparagraph\undefined\else
  \let\oldsubparagraph\subparagraph
  \renewcommand{\subparagraph}[1]{\oldsubparagraph{#1}\mbox{}}
\fi

\usepackage{color}
\usepackage{fancyvrb}
\newcommand{\VerbBar}{|}
\newcommand{\VERB}{\Verb[commandchars=\\\{\}]}
\DefineVerbatimEnvironment{Highlighting}{Verbatim}{commandchars=\\\{\}}
% Add ',fontsize=\small' for more characters per line
\usepackage{framed}
\definecolor{shadecolor}{RGB}{241,243,245}
\newenvironment{Shaded}{\begin{snugshade}}{\end{snugshade}}
\newcommand{\AlertTok}[1]{\textcolor[rgb]{0.68,0.00,0.00}{#1}}
\newcommand{\AnnotationTok}[1]{\textcolor[rgb]{0.37,0.37,0.37}{#1}}
\newcommand{\AttributeTok}[1]{\textcolor[rgb]{0.40,0.45,0.13}{#1}}
\newcommand{\BaseNTok}[1]{\textcolor[rgb]{0.68,0.00,0.00}{#1}}
\newcommand{\BuiltInTok}[1]{\textcolor[rgb]{0.00,0.23,0.31}{#1}}
\newcommand{\CharTok}[1]{\textcolor[rgb]{0.13,0.47,0.30}{#1}}
\newcommand{\CommentTok}[1]{\textcolor[rgb]{0.37,0.37,0.37}{#1}}
\newcommand{\CommentVarTok}[1]{\textcolor[rgb]{0.37,0.37,0.37}{\textit{#1}}}
\newcommand{\ConstantTok}[1]{\textcolor[rgb]{0.56,0.35,0.01}{#1}}
\newcommand{\ControlFlowTok}[1]{\textcolor[rgb]{0.00,0.23,0.31}{#1}}
\newcommand{\DataTypeTok}[1]{\textcolor[rgb]{0.68,0.00,0.00}{#1}}
\newcommand{\DecValTok}[1]{\textcolor[rgb]{0.68,0.00,0.00}{#1}}
\newcommand{\DocumentationTok}[1]{\textcolor[rgb]{0.37,0.37,0.37}{\textit{#1}}}
\newcommand{\ErrorTok}[1]{\textcolor[rgb]{0.68,0.00,0.00}{#1}}
\newcommand{\ExtensionTok}[1]{\textcolor[rgb]{0.00,0.23,0.31}{#1}}
\newcommand{\FloatTok}[1]{\textcolor[rgb]{0.68,0.00,0.00}{#1}}
\newcommand{\FunctionTok}[1]{\textcolor[rgb]{0.28,0.35,0.67}{#1}}
\newcommand{\ImportTok}[1]{\textcolor[rgb]{0.00,0.46,0.62}{#1}}
\newcommand{\InformationTok}[1]{\textcolor[rgb]{0.37,0.37,0.37}{#1}}
\newcommand{\KeywordTok}[1]{\textcolor[rgb]{0.00,0.23,0.31}{#1}}
\newcommand{\NormalTok}[1]{\textcolor[rgb]{0.00,0.23,0.31}{#1}}
\newcommand{\OperatorTok}[1]{\textcolor[rgb]{0.37,0.37,0.37}{#1}}
\newcommand{\OtherTok}[1]{\textcolor[rgb]{0.00,0.23,0.31}{#1}}
\newcommand{\PreprocessorTok}[1]{\textcolor[rgb]{0.68,0.00,0.00}{#1}}
\newcommand{\RegionMarkerTok}[1]{\textcolor[rgb]{0.00,0.23,0.31}{#1}}
\newcommand{\SpecialCharTok}[1]{\textcolor[rgb]{0.37,0.37,0.37}{#1}}
\newcommand{\SpecialStringTok}[1]{\textcolor[rgb]{0.13,0.47,0.30}{#1}}
\newcommand{\StringTok}[1]{\textcolor[rgb]{0.13,0.47,0.30}{#1}}
\newcommand{\VariableTok}[1]{\textcolor[rgb]{0.07,0.07,0.07}{#1}}
\newcommand{\VerbatimStringTok}[1]{\textcolor[rgb]{0.13,0.47,0.30}{#1}}
\newcommand{\WarningTok}[1]{\textcolor[rgb]{0.37,0.37,0.37}{\textit{#1}}}

\providecommand{\tightlist}{%
  \setlength{\itemsep}{0pt}\setlength{\parskip}{0pt}}\usepackage{longtable,booktabs,array}
\usepackage{calc} % for calculating minipage widths
% Correct order of tables after \paragraph or \subparagraph
\usepackage{etoolbox}
\makeatletter
\patchcmd\longtable{\par}{\if@noskipsec\mbox{}\fi\par}{}{}
\makeatother
% Allow footnotes in longtable head/foot
\IfFileExists{footnotehyper.sty}{\usepackage{footnotehyper}}{\usepackage{footnote}}
\makesavenoteenv{longtable}
\usepackage{graphicx}
\makeatletter
\def\maxwidth{\ifdim\Gin@nat@width>\linewidth\linewidth\else\Gin@nat@width\fi}
\def\maxheight{\ifdim\Gin@nat@height>\textheight\textheight\else\Gin@nat@height\fi}
\makeatother
% Scale images if necessary, so that they will not overflow the page
% margins by default, and it is still possible to overwrite the defaults
% using explicit options in \includegraphics[width, height, ...]{}
\setkeys{Gin}{width=\maxwidth,height=\maxheight,keepaspectratio}
% Set default figure placement to htbp
\makeatletter
\def\fps@figure{htbp}
\makeatother

\usepackage[noblocks]{authblk}
\renewcommand*{\Authsep}{, }
\renewcommand*{\Authand}{, }
\renewcommand*{\Authands}{, }
\renewcommand\Affilfont{\small}
\usepackage{mathtools}
\usepackage[left]{lineno}
\linenumbers
\usepackage[a4paper, total={6in, 10in}]{geometry}
\usepackage{longtable}
\usepackage{caption}
\usepackage[colorlinks=true,linkcolor=black,citecolor=black,urlcolor=black]{hyperref}
\usepackage{amsmath,amssymb,amsfonts,amsthm}
\usepackage{multirow}
\usepackage{setspace}\doublespacing
\renewcommand{\abstractname}{Summary}
\usepackage{bm}
\usepackage{algorithm}
\usepackage{algpseudocode}
\usepackage{rotating}
\usepackage{array}
\usepackage{doi}
\usepackage[sort, round]{natbib}
\usepackage{tikz}
\usepackage{float}
\usepackage{comment}
\usetikzlibrary{positioning}
\usetikzlibrary{shapes,decorations,arrows,calc,arrows.meta,fit,positioning}
\makeatletter
\makeatother
\makeatletter
\makeatother
\makeatletter
\@ifpackageloaded{caption}{}{\usepackage{caption}}
\AtBeginDocument{%
\ifdefined\contentsname
  \renewcommand*\contentsname{Table of contents}
\else
  \newcommand\contentsname{Table of contents}
\fi
\ifdefined\listfigurename
  \renewcommand*\listfigurename{List of Figures}
\else
  \newcommand\listfigurename{List of Figures}
\fi
\ifdefined\listtablename
  \renewcommand*\listtablename{List of Tables}
\else
  \newcommand\listtablename{List of Tables}
\fi
\ifdefined\figurename
  \renewcommand*\figurename{Figure}
\else
  \newcommand\figurename{Figure}
\fi
\ifdefined\tablename
  \renewcommand*\tablename{Table}
\else
  \newcommand\tablename{Table}
\fi
}
\@ifpackageloaded{float}{}{\usepackage{float}}
\floatstyle{ruled}
\@ifundefined{c@chapter}{\newfloat{codelisting}{h}{lop}}{\newfloat{codelisting}{h}{lop}[chapter]}
\floatname{codelisting}{Listing}
\newcommand*\listoflistings{\listof{codelisting}{List of Listings}}
\makeatother
\makeatletter
\@ifpackageloaded{caption}{}{\usepackage{caption}}
\@ifpackageloaded{subcaption}{}{\usepackage{subcaption}}
\makeatother
\makeatletter
\@ifpackageloaded{tcolorbox}{}{\usepackage[skins,breakable]{tcolorbox}}
\makeatother
\makeatletter
\@ifundefined{shadecolor}{\definecolor{shadecolor}{rgb}{.97, .97, .97}}
\makeatother
\makeatletter
\makeatother
\makeatletter
\makeatother
\ifLuaTeX
  \usepackage{selnolig}  % disable illegal ligatures
\fi
\usepackage[]{natbib}
\bibliographystyle{plainnat}
\IfFileExists{bookmark.sty}{\usepackage{bookmark}}{\usepackage{hyperref}}
\IfFileExists{xurl.sty}{\usepackage{xurl}}{} % add URL line breaks if available
\urlstyle{same} % disable monospaced font for URLs
\hypersetup{
  pdftitle={Documentation},
  pdfauthor={Kwaku Peprah Adjei},
  colorlinks=true,
  linkcolor={black},
  filecolor={Maroon},
  citecolor={black},
  urlcolor={black},
  pdfcreator={LaTeX via pandoc}}

\title{Documentation}
\author{Kwaku Peprah Adjei}
\date{}

\begin{document}
\maketitle
\ifdefined\Shaded\renewenvironment{Shaded}{\begin{tcolorbox}[sharp corners, boxrule=0pt, borderline west={3pt}{0pt}{shadecolor}, frame hidden, enhanced, breakable, interior hidden]}{\end{tcolorbox}}\fi

This document provides a walk-through on the model fitted for the
freshRestore project. The document begins with the objectives for the
modelling, dataset and covariates used and some results obtained.

\hypertarget{objectives}{%
\section{Objectives}\label{objectives}}

\begin{itemize}
\item
  Fit a population dynamic model (IPMs) with anthropogenic drivers
  \citep[human driven factors such as climate change, direct exploitation, pollution, biological invasions, sea-use change; ][]{moullec2021identifying}.
\item
  The IPM should be a mechanistic model
  \citep[plug and play;][]{frost2023integrated, smallegange2017mechanistic}.
  The IPMs should describe how the variables affect population.
\item
  IPMs should model temporal trends in human characteristics using
  variables that indicate catchment state of each year and a continuous
  time series data.
\item
  Produce prediction maps (of what?)
\end{itemize}

\hypertarget{example-dataset-received}{%
\section{Example dataset received}\label{example-dataset-received}}

I have received three files which are located
\href{https://github.com/Peprah94/fishyIPMs/tree/main/dataset}{here} in
the \href{https://github.com/Peprah94/fishyIPMs/tree/main}{gitHub
repository}:

\begin{itemize}
\item
  \emph{Example1.csv} : This file contains fish population data across
  lakes in Norway. The description of the column names in this file can
  be found
  \href{https://github.com/Peprah94/fishyIPMs/blob/main/dataset/Variable\%20description.docx}{here}.
\item
  \emph{Example2\_vatnLnr.csv} : This file contains a individual fish
  data of the population harvested. The description of the column names
  in this file can be found
  \href{https://github.com/Peprah94/fishyIPMs/blob/main/dataset/Variable\%20description.docx}{here}.
\item
  \emph{model\_catchment\_vars.csv} : This file contains the catchment
  variables for all the lakes. The description of the variables can be
  found
  \href{https://github.com/Peprah94/fishyIPMs/blob/main/meetingScripts/Description\%20of\%20catchment\%20variables\%20.pdf}{here}.
\end{itemize}

\hypertarget{data-format-and-exploration}{%
\section{Data format and
exploration}\label{data-format-and-exploration}}

The individual fish data was formatted into an age at harvest format
with their \textbf{lake}, \textbf{sex} and \textbf{year growth occurred}
as variables to widen the long-format dataset received.

\hypertarget{catchment-variables}{%
\subsection{Catchment Variables}\label{catchment-variables}}

\begin{itemize}
\item
  Easily accessible variables e.g.~using remote censored data
\item
  environmental covariates
\end{itemize}

I first perform a simple exploration of the catchment variables using
\emph{GGally} package. This is to help me reduce the number of catchment
variables to include in the model. But since the end product expected is
a plug and play type, I would have to include these catchment variables
of interest even if they are not selected in the pre-screening phase.

\begin{Shaded}
\begin{Highlighting}[]
\CommentTok{\# Not run}
\CommentTok{\# copy and run if needed}
\NormalTok{GGally}\SpecialCharTok{::}\FunctionTok{ggpairs}\NormalTok{(dataList}\SpecialCharTok{$}\NormalTok{ageAtHarvestData[, }\FunctionTok{c}\NormalTok{(}\DecValTok{53}\SpecialCharTok{:}\DecValTok{68}\NormalTok{)])}
\end{Highlighting}
\end{Shaded}

The following catchment variables are selected to model the fish
survival in the lakes:

\begin{itemize}
\item
  Coniferous forest
\item
  Moors and heathland
\item
  Peat bogs
\item
  Water bodies
\item
  Broad-leaved forest
\item
  Sparsely vegetated areas
\item
  mean\_ndvi
\end{itemize}

\hypertarget{other-covariates}{%
\subsection{Other covariates}\label{other-covariates}}

The \textbf{length at age this year} of the individual fishes was
modelled with the following covariates:

\begin{itemize}
\item
  Mean temperature in June
\item
  Precipitation in May
\item
  Precipitation in June
\item
  Precipitation in July
\end{itemize}

\begin{Shaded}
\begin{Highlighting}[]
\CommentTok{\# Not run}
\CommentTok{\# copy and run if needed}
\NormalTok{GGally}\SpecialCharTok{::}\FunctionTok{ggpairs}\NormalTok{(ageAtHarvestData[, }\FunctionTok{c}\NormalTok{(}\DecValTok{1}\SpecialCharTok{:}\DecValTok{3}\NormalTok{, }\DecValTok{15}\NormalTok{, }\DecValTok{36}\SpecialCharTok{:}\DecValTok{45}\NormalTok{)])}
\end{Highlighting}
\end{Shaded}

\hypertarget{fitted-model}{%
\section{Fitted model}\label{fitted-model}}

I fit a Bayesian age-at-harvest population model following
\cite {skelly2023flexible, baerum2021population}. The script to run for
the analysis can be found
\href{https://github.com/Peprah94/fishyIPMs/blob/main/ModelFitting/modelFit.R}{here}.

Here, I present a brief overview of the model fitted.

Let:

\begin{itemize}
\item
  \(i\) = individual
\item
  \(a\) = age under consideration. Here the age is from \(0\) to \(10\).
\item
  \(L_i\) = length at age this year for individual \(i\)
\item
  \(S_{ia}\) = sprawning of individual \(i\) at age \(a\)
\item
  \(F_{ia}\) = Fecundicity of individual \(i\) at age \(a\)
\end{itemize}

\hypertarget{modelling-l_i}{%
\subsection{\texorpdfstring{Modelling
\(L_i\)}{Modelling L\_i}}\label{modelling-l_i}}

\[
\begin{split}
L_i &\sim N(\lambda_L, \sigma^{2}_{L}) \\
\lambda_L &= X_L^T \beta_L
\end{split}
\] where \(X_L\) is a matrix of the covariates: Mean temperature in
June, Precipitation in May, Precipitation in June and Precipitation in
July; and \(\beta_L\) is a vector of covariate effect plus an intercept
term and \(\sigma^{2}_{L}\) is the variation of the error term.

The prior distributions assumed for \(\beta_L\) and \(\sigma^{2}_{L}\)
are \(N(0, 10^2)\) and \(Uniform(0.1, 10)\) respectively.

\hypertarget{sprawing-probability}{%
\subsection{Sprawing probability}\label{sprawing-probability}}

\[
\begin{split}
S_{ia} &\sim Bernouli(\psi_{ia}) \\
logit(\psi_{ia}) &= X_{sp}^T \beta_{sp} + \Gamma_a + \xi_{sex[i]}\\
\Gamma_a &\sim N(0, \sigma_a^2)\\
\xi_{sex[i]} &\sim N(0, \sigma_{sex}^2)
\end{split}
\] where \(X_{sp}\) is a matrix of covariates: age at year, \(L_i\), age
at year * \(L_i\), capture per unit effort; and \(\beta_{sp}\) is a
vector of covariate effect plus an intercept term.

\hypertarget{fecundicity}{%
\subsection{Fecundicity}\label{fecundicity}}

The equation for the fecundicity is derived from
\cite{baerum2021population}.

\[
F_{ia} = exp(log(L_i * 2.21 - 6.15) \times S_{ia})
\]

\hypertarget{survival-probability}{%
\subsection{Survival Probability}\label{survival-probability}}

\[
\begin{split}
logit(\gamma_{ia}) &= X_{surv}^T \beta_{surv} + \nu_a + \Xi_{lake[i]}\\
\nu_a &\sim N(0, \sigma_a^2)\\
\xi_{lake[i]} &\sim N(0, \sigma_{lake}^2)
\end{split}
\] where \(X_{surv}\) is a matrix of covariates described in section
3.1; and \(\beta_{surv}\) is a vector of covariate effect plus an
intercept term.

\hypertarget{projection-matrix-a}{%
\subsection{Projection Matrix (A)}\label{projection-matrix-a}}

The projection matrix used is the same as described in
\cite{baerum2021population}. The reader is referred to page 9 of the
\href{https://www.nature.com/articles/s41598-021-94350-x}{paper}.

\hypertarget{model-for-observed-counts-at-each-age}{%
\subsection{Model for observed counts at each
age}\label{model-for-observed-counts-at-each-age}}

The age at harvest count is modeled using the model described in
\citep{skelly2023flexible}. The change we make is to estimate the stable
population \(\lambda\) from the largest eigenvalue of the projection
matrix (A) instead of simulating it from an informed prior.

\hypertarget{results}{%
\section{Results}\label{results}}

I am currently running the model with \(20000\) iteration for each of
the \(2\) MCMC chains. It seems to take a while.


  \bibliography{references.bib}


\end{document}
